\newcommand{\code}{\texttt}

\documentclass{beamer}
%\usetheme{default}
%\usetheme{Boadilla}
%\usetheme{Madrid}
\usetheme{Pittsburgh}
%\usetheme{Rochester}
%\usetheme{Copenhagen}
%\usetheme{Warsaw}
%\usetheme{Singapore}
%\usetheme{Malmoe}
%\usecolortheme{albatross}

\usepackage{pseudocode}
\usepackage{listings}

\title[RCImmix]{A Description of the RCImmix Algorithm}
\subtitle[RC]{Reference Counting with better heap allocation}
\author[N. Jervis]{Nathan Jervis}
\institute[McMaster]{
  Department of Computer Science\\
  McMaster University, Hamilton\\
  \texttt{jervisnd@mcmaster.ca}\\
  \texttt{1211159}
}


\begin{document}


%--- the titlepage frame -------------------------%
\begin{frame}[plain]
  \titlepage
\end{frame}

%--- the presentation begins here ----------------%
\begin{frame}{Overview}
	\begin{itemize}
		\item Introduction to automatic memory management
		\item Problems with existing reference counting
		\item Optimizations to reference counting
		\item RCImmix algorithm
	\end{itemize}
\end{frame}

\begin{frame}{Manual Memory Management}
	\textbf{Manual Memory Management}
	\begin{itemize}
		\item Difficult to use
		\item Can cause dangling pointers
		\item Leads to memory leaks
	\end{itemize}
	\emph{Much better if the compiler/runtime can manage memory for us}
\end{frame}

\begin{frame}{Automatic Memory Management}
	\textbf{Tracing Garbage Collector:}
	\begin{itemize}
		\item Periodically pause program and follow program references
		\item Collect anything not referred to
	\end{itemize}
	\textbf{Reference Counting:}
	\begin{itemize}
		\item Counter keeps track of how many things are pointing to it
		\item When counter reaches 0, free memory
		\item \code{Allocate}, \code{Retain}, \code{Release}
	\end{itemize}
\end{frame}

\begin{frame}{Reference Counting vs Tracing}
	\textbf{Tracing Garbage Collector:}
	\begin{itemize}
		\item Little work for allocation
		\item Better cache performance (with compacting)
		\item Requires pausing to collect
	\end{itemize}
	\textbf{Reference Counting:}
	\begin{itemize}
		\item Doesn't pause
		\item Huge overhead
		\item Poor cache locality
	\end{itemize}
\end{frame}

\begin{frame}{RCImmix Optimizations}
	\textbf{Optimizations}
	\begin{itemize}
		\item Tracing collector as a backup
		\item Limited Bit Count
		\item Block Based allocation
		\item Nursery Allocation and Copying Collectors
		\item Objects are born as dead
	\end{itemize}
\end{frame}

\begin{frame}{Tracing Backup Collector}
	\begin{itemize}
		\item Naive reference counting won't collect cycles
		\item Reference counting handle normal memory collection
		\item When cycle garbage accumulates too much, call Tracer
		\item Tracer will very rarely be called, so you still don't have to worry about pauses too much
	\end{itemize}
\end{frame}

\begin{frame}{Limited bit count}
	\begin{itemize}
		\item In theory everything could point to object, requires 32/64 bits
		\item In practice most objects only a few things pointing to it
		\item Using 3-4 bits is fine
		\item If it overflows, just leave it at max, don't decrement
		\item Tracer will fix it when it collects
	\end{itemize}
\end{frame}

\begin{frame}{Block Based Allocation}
	\begin{itemize}
		\item Items are allocated out of a block
		\item Block keeps a pointer to the next free spot
		\item Count of number of live objects on the block
		\item Great for cache performance
	\end{itemize}
\end{frame}

\begin{frame}{Nurseries and Copying Collection}
	\begin{itemize}
		\item Objects are allocated into a nursery
		\item Move when mature (copied or pass collection cycle)
		\item Only new objects in nursery
			\begin{itemize}
				\item Will be collected during collection cycles
				\item Most objects die young, not copied
				\item Nurseries are cheap way to collect
			\end{itemize}
	\end{itemize}
\end{frame}

\begin{frame}{Objects are born as dead}
	\begin{itemize}
		\item When you create the object, garbage collector already considers it dead
		\item Only when it moves or matures do you consider it alive
		\item ModBuffer contains all objects that have created new objects
		\item Can process ModBuffer to check if "dead" objects are actually alive
	\end{itemize}
\end{frame}

\addtocounter{section}{1}


\begin{frame}{Typical Use}
\begin{pseudocode}{User Code}{args}
\MAIN
	x \GETS \CALL{Allocate}{size}\\
	\ldots\\
	\CALL{Retain}{x}\\
	y \GETS x\\
	\ldots\\
	\CALL{Release}{y}\\
	\ldots\\
	\CALL{release}{x}\\
\ENDMAIN
\end{pseudocode}
\end{frame}

\addtocounter{section}{1}

\begin{frame}
\begin{pseudocode}{Allocate}{size}
	\GLOBAL{bumpPointer, block}\\\\
	\IF size < block.end - bumpPointer \THEN
		\BEGIN
			pointer \GETS bumpPointer\\
			bumpPointer \GETS bumpPointer + size + 1\\
		\END\\
	\ELSE
		\BEGIN
			block \GETS \CALL{GetNewFreeBlock}{}\\
			\RETURN{\CALL{Allocate}{size}}\\
		\END\\\\
	pointer.new \GETS \TRUE
	pointer.count \GETS 0
	\RETURN{pointer}
\end{pseudocode}
\end{frame}


\begin{frame}
\begin{pseudocode}{Retain}{object}
	\IF object.new = \FALSE \THEN
	\BEGIN
		\IF object.count \NOT max \THEN
		object.count \GETS object.count + 1\\
		\RETURN{}
	\END\\
	pointer \GETS \CALL{CopyToNewLocation}{object}\\
	\CALL{AddToModBuffer}{object}
	object.count \GETS 2\\
	object.block.liveCount \GETS object.block.liveCount + 1
	\RETURN{}\\
\end{pseudocode}
\end{frame}

\begin{frame}
\begin{pseudocode}{Release}{object}
	\IF object.new = \TRUE \THEN
		\RETURN{}\\
	object.count \GETS object.count - 1\\
	\IF object.count = 0 \THEN
	\BEGIN
		\CALL{ProcessModBuffer}{}\\
		object.block.liveCount \GETS object.block.liveCount - 1\\
		\CALL{FreeBlocks}{}\\
	\END
\end{pseudocode}
\end{frame}

\addtocounter{section}{1}

\begin{frame}
\begin{pseudocode}{AddToModBuffer}{object}
	\GLOBAL{ModBuffer}\\
	\CALL{ModBuffer.Push}{object}
\end{pseudocode}
\end{frame}

\begin{frame}
\begin{pseudocode}{ProcessModBuffer}{none}
	\GLOBAL{ModBuffer}\\
	\FOREACH obj \in ModBuffer \DO
	\BEGIN
		FOREACH child \in obj.references \DO
		\BEGIN
			\IF child.new = \TRUE \THEN 
			\BEGIN
				\CALL{Retain}{child}
				child.count \GETS 1
			\END
		\END
	\END
\end{pseudocode}
\end{frame}
	

\end{document}