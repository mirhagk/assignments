\newcommand{\code}{\texttt}

\documentclass{beamer}
%\usetheme{default}
%\usetheme{Boadilla}
%\usetheme{Madrid}
\usetheme{Pittsburgh}
%\usetheme{Rochester}
%\usetheme{Copenhagen}
%\usetheme{Warsaw}
%\usetheme{Singapore}
%\usetheme{Malmoe}
%\usecolortheme{albatross}

\usepackage{listings}

\title[RCImmix]{A Description of the RCImmix Algorithm}
\subtitle[RC]{Reference Counting with better heap allocation}
\author[N. Jervis]{Nathan Jervis}
\institute[McMaster]{
  Department of Computer Science\\
  McMaster University, Hamilton\\
  \texttt{jervisnd@mcmaster.ca}
}


\begin{document}


%--- the titlepage frame -------------------------%
\begin{frame}[plain]
  \titlepage
\end{frame}

%--- the presentation begins here ----------------%
\begin{frame}{Overview}
	\begin{itemize}
		\item Introduction to automatic memory management
		\item Problems with existing reference counting
		\item The RCImmix algorithm
	\end{itemize}
\end{frame}


\begin{frame}[fragile]
\frametitle{Memory Management}
%{Memory Management}[fragile]


\begin{lstlisting}[caption=First C example]
int main()
{
    printf("Hello World!");
    return 0;
}
\end{lstlisting}

A displayed formula:

\[
  \int_{-\infty}^\infty e^{-x^2} \, dx = \sqrt{\pi}
\]

An itemized list:

\begin{itemize}
  \item itemized item 1
  \item itemized item 2
  \item itemized item 3
\end{itemize}

\begin{theorem}
  In a right triangle, the square of hypotenuse equals
  the sum of squares of two other sides.
\end{theorem}

\end{frame}

\end{document}