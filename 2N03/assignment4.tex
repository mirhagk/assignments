\documentclass[12pt]{article}
%\usepackage{natbib}
\usepackage{hyperref}
\usepackage{amsmath}
\usepackage[margin=1.1in]{geometry}

\title{\vspace{-2.5cm}Econ 2N03 - Problem 4\\Telgers}

\author{Nathan Jervis - 1211159}

\newcommand\act[1]{\textbf{#1}}
\newcommand\bold[1]{\textbf{#1}}
\newcommand\explain[1]{\text{  --  #1}}
\newcommand\italics[1]{\textit{#1}}
\long\def\/*#1*/{}

\begin{document}

\maketitle

%\vspace{-1.2cm}

With 90\% of wireless subscribers among 3 companies\cite{subStats}, Canadian wireless communications industry is a market with very little competition. If Rogers wanted to purchase Telus, it'd need to make a strong case to the Competition Bureau as to why it should allow a merger.

One of the key considerations when the Competition Bureau decides whether a merger can go through is the market shares of the respective companies\cite[VII 92.1]{compAct}. 

As of Q4 2013 Rogers has 9.5 million subscribers, \cite[Wireless - Subscriber Results]{rogersSub}, including those from all of it's subsidaries. Telus and bell both come in at approximately 7.8 million subscribers, with telus having a slight edge.\cite{telusSub}\cite{bellSub}\cite{subStats}. Overall there are 27.5 million wireless subscribers\cite{subStats}. This means Rogers commands 34.5\%, with Bell and Telus each having 28.4\%. Telus has also recently aquired Public Mobile\cite{telPublic}, which brings it's total subscriber base to nearly 30\%.

If Rogers and Telus were to merge at this point, together they'd command almost 65\% of the total market share. This would leave severely limit the competition in the market.

Another major concern is the subsidaries each company owns. Telus owns Koodo, Mike, Clearnet and is in the process of purchasing Public Mobile. Rogers owns Fido and Chatr. With a single owner, all these brands would appear to give consumers competition in a near monopolistic industry. In light of this, the Competition Bureau may decide that the confusion to consumers would be harmful, and disallow the merger on that point.

In 1998, Bank of Montreal proposed merging with Royal Bank at the same time that CIBC proposed merging with Toronto-Dominion.\cite[250]{bankMergers} The Competition Bureau denied the proposals\cite{letterBanks}. Among the reasons why, it listed concerns about competition of branches in several communities and reduced competition in the credit card industry.\cite[Summary of Conclusions A \& B]{letterBanks}

This sets a precedent for a response from the Competition Bureau in this situation. In light of the fact that they were concerned that the merging of 5 major banks into 3 major banks would reduce competition too much to allow the mergers, it seems likely that they would also deny a merger of the 2 largest wireless carriers to leave only 2 major wireless carriers.

\begin{thebibliography}{1}

  \bibitem{compAct}
\bold{Competition Act}\\
Canadian Government\\
R.S.C., 1985, c. C-34\\
URL: http://www.laws.justice.gc.ca/eng/acts/C-34/index.html

\bibitem{subStats}
\bold{Subscribers Stats}\\
Canadian Wireless Telecommunications Association\\
Q4 2013
URL: http://cwta.ca/wordpress/wp\-content/uploads/2011/08/SubscribersStats\_en\_2013\_Q4.pdf

\bibitem{rogersSub}
\bold{Rogers Communications Reports Fourth Quarter 2013 Results}\\
Rogers Communications Inc.\\
Subscriber results\\
URL: http://www.newswire.ca/en/story/1304367/rogers-communications-reports-fourth-quarter-2013-results

\bibitem{telusSub}
\bold{TELUS ended 2013 with 7,807,000 wireless subscribers...}
Ian Hardy\\
Mobile Syrup\\
URL: http://mobilesyrup.com/2014/02/13/telus-ended-2013-with-7807000-wireless-subscribers-77-of-postpaid-subs-have-a-smartphone/

\bibitem{bellSub}
\bold{Bell reports Q4 and 2013 results: 7,778,334 subs...}
Ian Hardy\\
Mobile Syrup\\
URL: http://mobilesyrup.com/2014/02/06/bell-reports-q4-and-2013-results-7778334-subs-arpu-of-57-25-and-73-now-use-a-smartphone/

\bibitem{bankMergers}
\bold{Changing Canada: Political Economy as Transformation}\\
Leah F. Vosko\\
McGill-Queens Press

\bibitem{letterBanks}
\bold{ Competition Bureau's Letter to the Toronto-Dominion Bank and Canada Trust}\\
Commissioner of Competition\\
Competition Bureau\\
January 28, 2000\\
URL: http://www.competitionbureau.gc.ca/eic/site/cb-bc.nsf/eng/01649.html
  \end{thebibliography}

\end{document}