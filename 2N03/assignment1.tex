\documentclass{article}
\usepackage{natbib}
\usepackage{hyperref}

\title{Econ 2N03 - Problem 1\\Not all monopoly is bad}

\author{Nathan Jervis - 1211159}

\newcommand\act[1]{\textbf{#1}}

\begin{document}

\maketitle

Electric energy distribution in Ontario is handled through several local monopolies. Each region has a different company, and this company has a complete monopoly over distributing electricity in the area.

There is a lot of discussion about whether these monopolies should be enforced and regulated by the government. Some proponents argue that it'd be wasteful to have 2 competing power lines in the same area. The argument goes that since having a single line costs less, a single company owning those lines would be more efficient than multiple companies, and the lower costs would lead to lower prices.

The problem is that a monopoly can choose to set it's price to maximum profit, and is not forced into more competitive prices by other companies. In such a scenario, the lower costs may not have an affect on the output cost, which is determined based mostly on demand. In a relatively inelastic market, the situation worsens even more as the demand doesn't decrease with higher prices.

Within the mobile telecommunications industry, a natural ogliopoloy has existed with Rogers, Bell and Telus for many years. Each company owns a re-branded child company to appear to have more competition (Fido,Virgin and Koodo respectively), but in actuality these three companies have traditionally had equivalent prices, and demanded high prices for the service.

In July 2008, the government decided to reserve sections of the wireless spectrum for sale only to new companies, helping new companies get into the market. As a result of this, new mobile companies opened up. None of them had the money to build a complete infrastructure, but they all built within particular areas, and rented out lines of other companies.

This allowed them to offer consumers a low price within their own network, while paying the old premium price when it required use of other companies networks.

The CRTC monitors prices of all telecommunications services, and publishes papers with the results. For mobile phones, they have 3 plans that they check the prices of, a low tier, medium tier and high tier. They then report the average price based on the companies in existence, and since 2008 the prices have gone done, especially at the higher tiers (tier 2 went from \$60.81 to \$44.86).

Natural monopolies may form with some utilities based on cost of entering, but whenever the gap between costs and revenue becomes high enough to offset the cost of building a competing network, a new competitor may join the market, bringing prices down to a more reasonable rate. This is what happened with the mobile phone industry.

If a monopoly is government owned, or regulated in some way to control the negative effects, in can be successful, taking advantage of lower costs. Left to run on their own though, a monopoly's low cost makes little difference in the price they charge consumers, and allowing or encouraging competition can bring the costs to a more reasonable rate.

\url{http://www.ic.gc.ca/eic/site/oca-bc.nsf/eng/ca02348.html}


\url{http://laws.justice.gc.ca/eng/acts/T-3.4/page-1.html#h-1}

\url{http://www.crtc.gc.ca/eng/publications/reports/rp130422.pdf}

\url{https://www.ieso.ca/imoweb/pubs/local_distribution/Ontario_LDC_Map.pdf}

%\cite{nigeria}

%\cite{ahu61}

\bibliographystyle{te}

\bibliography{reference}

\end{document}