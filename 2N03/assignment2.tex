\documentclass{article}
\usepackage{natbib}
\usepackage{hyperref}

\title{\vspace{-3.5cm}Econ 2N03 - Problem 1\\I'll talk to the Competition Bureau}

\author{Nathan Jervis - 1211159}

\newcommand\act[1]{\textbf{#1}}

\begin{document}

\maketitle

\vspace{-1.2cm}
\section{Summary}

What Rebecca is proposing in this scenario is price fixing. Canadian law is quite clear on the subject: \quote{Every person commits an offence who, with a competitor of that person with respect to a product, conspires, agrees or arranges to fix, maintain, increase or control the price for the supply of the product;}{Competition Act 45.1 a)}

In this situation, Rebecca is approaching Sharon stating that if Sharon agrees, Rebecca will fix her prices at the current level. If Sharon agrees, then the two companies join a cartel, and reap the benefits assoicated with monopolies.

McAfee and McMillian (bidding rings) argue that any successful cartel has 4 obstacles to overcome. 

1. A way to divvy up the spoils (choose who get which contracts)
2. A way to enforce the agreement (contracts aren't possible due to it being illegal, so another means must be discovered)
3. A way to prevent new firms from joining (high profits means that it's a more attractive business to new companies)
4. A way to prevent the victims from stopping the cartel

If Sharon was to agree to the deal, these 4 obstacles must be overcome. The first two obstacles are both related to trust and greed, and not knowing anything about the individuals \foonote{other than that Rebecca is either willing to break the law, or does not understand it}, it's difficult to assess these, so these will be ignored. The third obstacle is the more difficult to overcome in this scenario. Many 


http://www.jstor.org.libaccess.lib.mcmaster.ca/stable/2117323?seq=1&uid=3739448&uid=2&uid=3737720&uid=4&sid=21103801816521&__redirected


Electric energy distribution in Ontario is handled through several local monopolies. Each region has a different company, and this company has a complete monopoly over distributing electricity in the area.

There is a lot of discussion about whether these monopolies should be enforced and regulated by the government. Some proponents argue that it'd be wasteful to have 2 competing power lines in the same area. The argument goes that since having a single line costs less, a single company owning those lines would be more efficient than multiple companies, and the lower costs would lead to lower prices.

The problem is that a monopoly can choose to set it's price to maximum profit, and is not forced into more competitive prices by other companies. In such a scenario, the lower costs may not have an affect on the output cost, which is determined based mostly on demand. In a relatively inelastic market, the situation worsens even more as the demand doesn't decrease with higher prices.

Within the mobile telecommunications industry, a natural ogliopoloy has existed with Rogers, Bell and Telus for many years. Each company owns a re-branded child company to appear to have more competition (Fido,Virgin and Koodo respectively), but in actuality these three companies have traditionally had equivalent prices, and demanded high prices for the service.

In July 2008, the government decided to reserve sections of the wireless spectrum for sale only to new companies, helping new companies get into the market. As a result of this, new mobile companies opened up. None of them had the money to build a complete infrastructure, but they all built within particular areas, and rented out lines of other companies.

This allowed them to offer consumers a low price within their own network, while paying the old premium price when it required use of other companies networks.

The CRTC monitors prices of all telecommunications services, and publishes papers with the results. Since 2008 the prices have gone done, especially at the higher levels (level 2 went from \$60.81 to \$44.86).

Natural monopolies may form with some utilities based on cost of entering, but whenever the gap between costs and revenue becomes high enough to offset the cost of building a competing network, a new competitor may join the market, bringing prices down to a more reasonable rate. This is what happened with the mobile phone industry.

If a monopoly is government owned, or regulated in some way to control the negative effects, in can be successful, taking advantage of lower costs. Left to run on their own though, a monopoly's low cost makes little difference in the price they charge consumers, and allowing or encouraging competition can bring the costs to a more reasonable rate.

\section{References}

"Cellphone Services — Recent Consumer Trends." Government of Canada, Industry Canada, National Capital Region, Office of the Deputy Minister, Senior Associate Deputy Minister, Office of Consumer Affairs.\\ http://www.ic.gc.ca/eic/site/oca-bc.nsf/eng/ca02348.html (accessed May 13, 2014).\\

"Telecommunications Act (S.C. 1993, c. 38)." Legislative Services Branch. http://laws.justice.gc.ca/eng/acts/T-3.4/page-1.html\#h-1 (accessed May 13, 2014).\\

Wall Communications Inc.\textit{Price Comparisons of Wireline, Wireless and Internet Services in Canada and with Foreign Jurisdictions} Canadian Radio-television and Telecommunications Commission and Industry Canada , 2013\\

"ONTARIO’S ELECTRICITY DISTRIBUTION SYSTEM." ieso. 
 \\https://www.ieso.ca/imoweb/pubs/local\_distribution/Ontario\_LDC\_Map.pdf (accessed May 13, 2014).\\

DiLorenzo, Thomas J.. "The Myth Of Natural Monopoly." The Review of Austrian Economics 9, no. 2 (1996): 43-58.\\

%\url{http://www.ic.gc.ca/eic/site/oca-bc.nsf/eng/ca02348.html}

%\url{http://laws.justice.gc.ca/eng/acts/T-3.4/page-1.html#h-1}

%\url{http://www.crtc.gc.ca/eng/publications/reports/rp130422.pdf}

%\url{https://www.ieso.ca/imoweb/pubs/local_distribution/Ontario_LDC_Map.pdf}

%\cite{nigeria}

%\cite{ahu61}

%\bibliographystyle{te}

%\bibliography{reference}

\end{document}