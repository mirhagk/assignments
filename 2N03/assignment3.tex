\documentclass[12pt]{article}
%\usepackage{natbib}
\usepackage{hyperref}
\usepackage{amsmath}
\usepackage[margin=1.0in]{geometry}

\title{\vspace{-3.5cm}Econ 2N03 - Problem 3\\It isn't worth it}

\author{Nathan Jervis - 1211159}

\newcommand\act[1]{\textbf{#1}}
\newcommand\bold[1]{\textbf{#1}}
\newcommand\explain[1]{\text{  --  #1}}
\newcommand\italics[1]{\textit{#1}}
\long\def\/*#1*/{}

\begin{document}

\maketitle

%\vspace{-1.2cm}

The HP EliteBook Folio is an overpriced laptop. This can be determined by observing \italics{Everything Computers} actions.

Edward\cite{baitSwitch} has laid out conditions for an equilibrium in which bait and switch tactics are employed by sellers. One important condition is that more people want the i3 HPs priced at \$400 than the i7 HPs priced at \$1500.\cite[pg. 818]{baitSwitch}. This gives us a clear picture of the market, and the comparatively lackluster demand for the i7 HPs.

We know that more consumers prefer the i3 over the i7 at their respective price points. We also know that \italics{Everything} is willing to risk a potentially hefty fine. The Competition Act, section 74.04 clearly states that the company must provide enough of a supply of the advertised model, and failure to do so is an offense. Since Mike arrived a few minutes after opening \cite[para 3]{problem} the store clearly didn't have enough to supply the demand. The potential fine is up to \$10 000 000\cite[74.1]{compAct}.

If the model shown in the picture was identifiable as the i7 model, and not the i3 model, then the company is also in violation of section 52, which carries a potentially unlimited fine, and up to 14 years in prison.\cite[52.5.a]{compAct} Both of these fines are unlikely to be this high, but they could still be substantial.

The cost of risking the fine be recuperated in the difference in value to consumers. Profit differences between the i3 and the i7 must be large to be worth it to \italics{Everything}.

These fines exist because of the waste to consumer's time, and because in many cases consumers would end up paying anywhere from double to nine times the price that was advertised\cite[pg. 274]{badBait}

If they had simply placed the i7 ad, they would not see as much profit\footnote{Otherwise they would have placed the i7 as the one in the advertisement}, and selling the i3 would not as much profit either. Therefore the profit margin on the i7 must be larger than the i3, and big enough to take these risks. 

If Mike knows his rights, then he should mention to the salesperson section 74.04 of the Consumer Protection Act. Most likely the store will offer him the i3 at the advertised price on backlog, or something equivalent, since it isn't worth it to them to fight it. Knowing your rights as a consumer can ensure that situations like this don't happen. \italics{Everything Computers} is counting on the majority of consumers to not know their rights.

\begin{thebibliography}{1}

  \bibitem{compAct}
\bold{Competition Act}\\
Canadian Government\\
R.S.C., 1985, c. C-34\\
URL: http://www.laws.justice.gc.ca/eng/acts/C-34/page-27.html\#h-20

\bibitem{baitSwitch}
\bold{Bait And Switch}\\
Lazear, Edward P\\
The University of Chicago Press\\
Journal of Political Economy\\
Vol. 103, No. 4, Aug., 1995\\
Stable URL: http://www.jstor.org/stable/2138583

\bibitem{problem}
\bold{Problem Description}\\
Dr. Barb Bloemhof\\
June 2014\\
Problem 3\\
URL: https://avenue.cllmcmaster.ca/d2l/common/viewFile.d2lfile/Database/MjA0NDAyNQ/Problem\%203.pdf?ou=129181

\bibitem{badBait}
\bold{Can Bait and Switch Benefit Consumers?}\\
James D. Hess, James D. Hess\\
Volume 9 Issue 2, May 1990, pp. 114-124\\
URL: http://dx.doi.org/10.1287/mksc.9.2.114

  \end{thebibliography}

\end{document}